\begin{table}[h!]
\centering
\begin{tabular}{||c | c | c||}
 \hline
 Classifier & 100 test samples & 1000 test samples \\ [0.5ex] 
 \hline\hline
 Support Vector Machine & 100\% & 99.7\% \\ 
 \hline
 k-Nearest Neighbors & 99\% & 99.4\% \\
 \hline
 Random Forest & 99\% & 99.6\% \\
 \hline
 Logistic Regression & 100\% & 99.5\% \\
 \hline
 Naive Bayes & 100\% & 98.9\% \\ [1ex] 
 \hline
\end{tabular}
\caption{Comparison between accuracies of different classifiers using LBP feature and different sample size.}
\label{table:1}
\end{table}

\begin{table}[h!]
\centering
\begin{tabular}{||c | c||}
 \hline
 Component & Execution Time \\ [0.5ex] 
 \hline\hline
 Form Clipping & 0.09 \\ 
 \hline
 Line Segmentation & 0.06 \\
 \hline
 LBP Features & 0.12 \\
 \hline
 Classifier Training & 0.01 \\
 \hline
 Complete test case & 2.00 \\ [1ex] 
 \hline
\end{tabular}
\caption{Average execution time of different system components (measured in seconds).}
\label{table:2}
\end{table}

\subsection{Accuracy Analysis}
As mentioned before, different approaches are considered for both feature extraction and classification. \\
We start by examining different feature extractors. According to \cite{c1}, the most promising texture descriptors are \emph{LBP}, \emph{LPQ} and \emph{GLCM}. We tried these texture descriptors, however we find that \emph{LBP} offers the most accurate and fast results, so we considered it for further experimentation. \emph{LBP} offers $99\%$ average accuracy with all classifiers in \textbf{sampled train mode}, however \emph{LPQ} and \emph{GLCM} offers around $95\%$ and $90\%$, respectively. \\
Regarding the classifiers, table \ref{table:1} shows the accuracy of different classifiers using \emph{LBP} features on $100$ and $1000$ random test cases. We can see that \emph{SVM} and \emph{RF} offer comparable results, however we choose \emph{SVM}, as it is more robust to \emph{preprocessor} failures and offers more consistent accuracy.

\subsection{Time Analysis}
We try to maintain our system accuracy within reasonable execution time. Different components are implemented and optimized for time. The execution times for different components are shown in table \ref{table:2}.
