Writer identification from handwriting is a challenging problem. Historically, experts with domain knowledge were required to tackle such tricky problem. However, with the rise of \emph{AI} and \emph{machine learning} techniques, systems can be built to solve the handwriting identification problem. In \emph{machine learning} systems, the choice of good features and robust classifiers is the core challenge. For such problem, domain-based features can be used such as \emph{codebooks} and \emph{grapheme signatures}. However, with the evolution of general purpose texture descriptors like \emph{local binary pattern} and\emph{local phase quantization}, it turns out that these features can perform even better in most cases. For this reason, we decided to adopt the fast and well-known \emph{local binary pattern} texture descriptor, inspired by \cite{c1}. We, also, considered multiple classifiers and decided on \emph{support vector machine} classifier, which is the best in our case.